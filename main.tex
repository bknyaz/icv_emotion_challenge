\documentclass{article}
\usepackage[utf8]{inputenc}
\usepackage{url}
\title{Emotion challenge fact sheet}
\author{iCV}
\date{February 2017}

\begin{document}

\maketitle

\section{Team details}

\begin{itemize}
\item Team name: bknyaz                                  
\item Team leader name: Boris Knyazev                           
\item Team leader address, phone number and email: 

Moscow, Russia

\texttt{8e8all@gmail.com}, \texttt{borknyaz@gmail.com}

\item Rest of the team members: None                    
\item Team website URL (if any): \url{https://github.com/bknyaz}
\item Affiliation: NTechLab
\end{itemize}

\section{Contribution details}

\begin{itemize}
\item Title of the contribution: Simple unsupervised learning method
\item Final score: 86.77\%                                              
\item General method description: This result is obtained based on the model described in [1]. The model is a convolutional neural network, filters of which are trained layer-wise using k-means clustering, that is without backpropagation. Recursive autoconvolution applied to image patches before k-means improves results. A linear SVM is applied on top of extracted features. In this challenge, a committee of 5 simple models with a single convolutional layer of 512 filters turned out to work well.
\item References

[1] B. Knyazev, E. Barth and Thomas Martinetz, "Recursive Autoconvolution for Unsupervised Learning of Convolutional Neural Networks", accepted to IJCNN-2017
\item Representative image / diagram of the method        
\item Describe data preprocessing techniques applied (if any): faces are extracted and aligned using dlib (\url{https://github.com/davisking/dlib})
\end{itemize}


\section{Face Landmarks Detection}
\subsection{Features / Data representation}
Describe features used or data representation model FOR FACE LANDMARKS DETECTION (if any)

Face Landmarks are extracted by dlib, they are used to align faces

\subsection{Dimensionality reduction}
Dimensionality reduction technique applied FOR FACE LANDMARKS DETECTION (if any)

\subsection{Compositional model}
Compositional model used, i.e. pictorial structure FOR FACE LANDMARKS DETECTION (if any)

\subsection{Learning strategy}
Learning strategy applied FOR FACE LANDMARKS DETECTION (if any)

Publicly available Face Landmarks model is used: available at \url{http://dlib.net/files/shape_predictor_68_face_landmarks.dat.bz2}.

\subsection{Other techniques}
Other technique/strategy used not included in previous items FOR FACE LANDMARKS DETECTION (if any)

\subsection{Method complexity}
Method complexity FOR FACE LANDMARKS DETECTION


\section{Dominant emotion recognition}
\subsection{Features / Data representation}
Describe features used or data representation model FOR DOMINANT EMOTION RECOGNITION (if any)

\subsection{Dimensionality reduction}
Dimensionality reduction technique applied FOR DOMINANT EMOTION RECOGNITION (if any)

\subsection{Compositional model}
Compositional model used, i.e. pictorial structure FOR DOMINANT EMOTION RECOGNITION (if any)

\subsection{Learning strategy}
Learning strategy applied FOR DOMINANT EMOTION RECOGNITION (if any)

\subsection{Other techniques}
Other technique/strategy used not included in previous items FOR DOMINANT EMOTION RECOGNITION (if any)

\subsection{Method complexity}
Method complexity FOR DOMINANT EMOTION RECOGNITION


\section{Complementary emotion recognition}
\subsection{Features / Data representation}
Describe features used or data representation model FOR COMPLEMENTARY EMOTION RECOGNITION (if any)

\subsection{Dimensionality reduction}
Dimensionality reduction technique applied FOR COMPLEMENTARY EMOTION RECOGNITION (if any)

\subsection{Compositional model}
Compositional model used, i.e. pictorial structure FOR COMPLEMENTARY EMOTION RECOGNITION (if any)

\subsection{Learning strategy}
Learning strategy applied FOR COMPLEMENTARY EMOTION RECOGNITION (if any)

\subsection{Other techniques}
Other technique/strategy used not included in previous items FOR COMPLEMENTARY EMOTION RECOGNITION (if any)

\subsection{Method complexity}
Method complexity FOR COMPLEMENTARY EMOTION RECOGNITION


\section{Joint dominant and complementary emotion recognition}
\subsection{Features / Data representation}
Describe features used or data representation model FOR JOINT DOMINANT AND COMPLEMENTARY EMOTION RECOGNITION (if any)

\subsection{Dimensionality reduction}
Dimensionality reduction technique applied FOR JOINT DOMINANT AND COMPLEMENTARY EMOTION RECOGNITION (if any)

\subsection{Compositional model}
Compositional model used, i.e. pictorial structure FOR JOINT DOMINANT AND COMPLEMENTARY EMOTION RECOGNITION (if any)

\subsection{Learning strategy}
Learning strategy applied FOR JOINT DOMINANT AND COMPLEMENTARY EMOTION RECOGNITION (if any)

\subsection{Other techniques}
Other technique/strategy used not included in previous items FOR JOINT DOMINANT AND COMPLEMENTARY EMOTION RECOGNITION (if any)

\subsection{Method complexity}
Method complexity FOR JOINT DOMINANT AND COMPLEMENTARY EMOTION RECOGNITION


\section{Global Method Description}

\begin{itemize}
\item Total method complexity: all stages
\item Which pre-trained or external methods have been used (for any stage, if any) 
\item Which additional data has been used in addition to the provided training and validation data (at any stage, if any) 
\item Qualitative advantages of the proposed solution
\item Results of the comparison to other approaches (if any)
\item Novelty degree of the solution and if is has been previously published
\end{itemize}

\section{Other details}

\begin{itemize}
\item Language and implementation details (including platform, memory, parallelization requirements)
\item Detailed list of prerequisites for compilation
\item Human effort required for implementation, training and validation?
\item Training/testing expended time? 
\item General comments and impressions of the challenge?
\end{itemize}
\end{document}
